\documentclass{article}
\usepackage{graphicx} % Required for inserting images
\usepackage{geometry}
\usepackage{array}

\title{sample(Employee Records) -26.4.24}
\author{Hemalatha A}
\date{April 2024}

\begin{document}
\maketitle
\section{Employee Information}
\begin{table}[h]
\centering
\begin{tabular}{|c|c|c|c|}
\hline
\textbf{Name} & \textbf{Position} & \textbf{Contact Information} & \textbf{Performance Evaluation} \\
\hline
John Doe & Manager & john@example.com & Meets Expectations \\
\hline
Jane Smith & Sales Associate & jane@example.com & Exceeds Expectations \\
\hline
Michael Johnson & Developer & michael@example.com & Outstanding \\
\hline
\end{tabular}
\caption{Employee Records}
%assigning a label to the table
\label{tab:Employee Records}
\end{table}
Table \ref{tab:Employee Records} displays sample employee records. It includes columns for employee names, positions, contact information, and performance evaluations.

\section{Explanation of Components}
In the table above, we use several components to structure the data:
\begin{itemize}
  \item \texttt{\textbackslash hline}: Draws a horizontal line across the table.
  \item \texttt{\textbackslash textbf\{\}}: Makes the text bold.
  \item \texttt{\&}: Separates columns within a row.
  \item \texttt{\textbackslash\textbackslash}: Starts a new row.
  \item \texttt{\{c|c|c|c\}}: Specifies the alignment of each column. Here, "c" means center-aligned, and "|" adds vertical lines between columns.
  \item \texttt{[h]}: Suggests that the table should be placed approximately "here" in the document.
\end{itemize}

\section{Math Mode}
\label{sec:math}
Here is an example of an inline equation: $E=mc^2$.
And here is a displayed equation:
\[
    \int_{0}^{1} x^2 \, dx = \frac{1}{3}.
\]

\section{Figures}
\begin{figure}[h]
    \centering
    \includegraphics[width=0.5\textwidth]{example-image}
    \caption{An example image.}
    \label{fig:image}
\end{figure}

Figure \ref{fig:image} shows an example image.

\section{Headings}
\subsection{Subsection}
This is a subsection.

\subsubsection{Subsubsection}
This is a subsubsection.

\section{Cross-Referencing}
As mentioned in Section \ref{sec:math}, the formula $E=mc^2$ is fundamental.
Also, see Figure \ref{fig:image} for the example image.

\end{document}
